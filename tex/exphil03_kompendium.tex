\documentclass[a4paper]{IEEEtran}
\usepackage[T1]{fontenc}
\usepackage{lmodern}
\usepackage[utf8]{inputenc}
\usepackage[T1]{fontenc,url}
\usepackage[norsk]{babel}
\usepackage{amsmath}
\usepackage{amsfonts}
\usepackage{amsthm}
\usepackage{amssymb}
\usepackage[justification=centering]{caption}
\usepackage{listingsutf8}
\usepackage{url}
\usepackage{graphicx}
\usepackage{pdfpages}
\usepackage{tabularx}
\usepackage{hyperref}
\usepackage{pdfpages}
\usepackage{placeins}
%\usepackage{subcaption}
\usepackage{color}
\usepackage[square,numbers]{natbib}
\usepackage{upgreek}
\usepackage{float}
\usepackage{matlab-prettifier}
%\usepackage[justification=centering]{caption}
\usepackage{afterpage}

\bibliographystyle{abbrvnat}

\definecolor{mygreen}{rgb}{0,0.6,0}
\definecolor{mygray}{rgb}{0.5,0.5,0.5}
\definecolor{mymauve}{rgb}{0.58,0,0.82}
\definecolor{mylilas}{rgb}{255,0,0}

\hypersetup{
  colorlinks=true,
  linkcolor=blue,
  filecolor=magenta,
  urlcolor=cyan
}

\graphicspath{{../bileter/}}

\lstset{
    language=Matlab,%
    inputencoding=utf8/latin1,
    frame = shadowbox,
    basicstyle=\tiny,
    %style = Matlab-editor,
    %basicstyle=\color{red},
    breaklines=true,%
    morekeywords={matlab2tikz},
    keywordstyle=\color{blue},
    morekeywords=[2]{1}, keywordstyle=[2]{\color{black}},
    identifierstyle=\color{black},%
    stringstyle=\color{mylilas},
    commentstyle=\color{mygreen},%
    showstringspaces=false,%without this there will be a symbol in the places where there is a space
    numbers=left,
    numberstyle={\tiny \color{black}},% size/color of the numbers
    numbersep=9pt, % this defines how far the numbers are from the text
    emph=[1]{for,end,break,if,while,return},emphstyle=[1]\color{red}, %some words to emphasise
    %emph=[2]{word1,word2}, emphstyle=[2]{style},
}


\title{\bigskip EXPHIL03, vår 2017 \\[1cm] KOMPENDIUM \\[5cm]}

\author{Lars André Møen - github.com/7arsandre }
\begin{document}
\maketitle
\onecolumn
\tableofcontents
\clearpage
\twocolumn
\onecolumn

\section{Pensum}
\subsection{Litteratur}
    \begin{center}
        \textbf{Pensumlitteratur til examen philosopicum}\bigskip    
    \end{center}
    
    
    \textbf{Exphil 1 - Filosofi- og vitenskapshistorie}. 2015, Gyldendal Akademiske forlag\bigskip
 
    
    \textbf{Exphil 2 - Tekster i etikk}. 2015, Gyldendal Akademiske forlag\bigskip

    
    (Tidlegare utgåver kan også nyttast: \bigskip
    
    \textbf{Exphil 1 - Filosofi- og vitenskapshistorie} 6. utgåve 2013.
    
     IFIKK, Universitetet i Oslo.
     
     ISBN: 978-82-91670-58-4g\bigskip

    
    \textbf{Exphil 2 - Tekster i etikk}. 5. utgåve, 2013 eller tidlegare.
    
     IFIKK, Universitetet i Oslo.
     
     ISBN: 978-82-91670-57-7)
    
\subsection{Exphil 1}\bigskip
\begin{center}

\textbf{Exphil 1 pensum}\bigskip

\begin{tabularx}{0.75\textwidth}{| X | X | X |}\hline	
            Filosof
            &
            Kompendium ref.
            &
            Kjelde \\ \hline
            Platon 
            &
            \ref{platon}
            &
            kap. 3 - 4 og 20
            \\ \hline
            Aristoteles 
            &
            \ref{aristoteles}
            &
            kap. 5 - 8 og 21
            \\ \hline
            René Descartes 
            &
            \ref{descartes}
            &
            kap. 9 - 10 og 22
            \\ \hline
            David Hume 
            &
            \ref{hume}
            &
            kap. 11 - 12 og 23
            \\ \hline
            Immanuel Kant 
            &
            \ref{kant}
            & 
            kap, 13 -16 og 24
            \\ \hline
            Simone de Beauvoir 
            &
            \ref{beauvoir}
            & 
            kap. 17 - 19 og 25 - 27
            \\ \hline
    \end{tabularx}
\end{center}

\subsection{Exphil 2}\bigskip
\begin{center}

\textbf{Exphil 2 penusm}\bigskip

\begin{tabularx}{0.75\textwidth}{| X | X | X |}\hline	  
            Tema
            &
            Kompendium ref.
            &
            Kjelde \\ \hline
            Relativitisme 
            &
            \ref{relativisme}
            &
            kap. 1 - 3
            \\ \hline
            Utilitarisme 
            &
            \ref{utilitarisme}
            &
            kap. 5 - 6
            \\ \hline
            Deontologi
            &
            \ref{deontologi}
            &
            kap. 7
            \\ \hline
            Dydsetikk 
            &
            \ref{dydsetikk}
            &
            kap. 8
            \\ \hline
            Foster og dyrs moralske status, miljøetikk 
            &
            \ref{fdms}
            & 
            kap, 9 - 12
            \\ \hline
            Feministisk etikk 
            &
            \ref{femimisme}
            & 
            kap. 4
            \\ \hline
    \end{tabularx}
\end{center}\clearpage
\twocolumn
\section{Platon}
\label{platon}\bigskip

Født i døde i Athen. Kom tidleg under Sokrates påverknad. Platon brukte store delar av livet sitt til å undervisa. Han var sterkt opptatt av samfunnsproblemer og var til dels også aktiv politiker. Litteraturen hans består for det meste av dialoger (34 stk.) I dialogane er det tydeleg at han er påverka av Sokrates. I desse dialogane, som mange andre, er Sokrates hovudersonen og Platons talerøyr.\bigskip

Dualistisk verdsyn. Motstandar av sofisme.\bigskip

Litteratur:
\begin{itemize}
    \item Menon
    \item Staten
\end{itemize}\bigskip 

Sentralt ang. \textit{Platon}:
\begin{itemize}
    \bigskip
    
    \item Dyd
    \item Menon
    \item Sokrates var hans læremeister
    \item Sokratisk ironi
    \item Sokrates ser på seg sjølv som jordmor, der han er med på å få samtalepartnaren til å føde innsikt
    \item Slaveguten (frå teksten Menon)
    \item Gjenerindringslæra (læring kun gjenerindring av tidligere viten.Sjela er nemlig udødelig, og har blitt født mange gongar tidligare)
    \item Sann/usann oppfatning
    \item Den udødelige sjel
    \item Fornuft er betre kilde til kunnskap enn fornuft
    \item Meiner staten bør styres av filosofene (frå teksten Staten)
    \item Idealsamfunnet (frå teksten Staten)
    '\item Rasjonalist (\ref{rasjonalisme})
     
\end{itemize}\bigskip

\section{Aristoteles}
\label{aristoteles}\bigskip

Gresk filosof og naturforskar (384-322 f.Kr), elev av Platon. Ideer; Den gylne middelveg, fornuft, logikk, biologi og lidenskap. Grunnleggjande tekstar innan metafysikk, biologi, etikk, politisk filosofi, retorikk, poetikk, psykologi, logikk og vitenskapsfilosofi.
\bigskip

Litteratur:
\begin{itemize}
    \item Metafysikken
    \item Om sjelen
    \item Den nikomakiske etikk
    \item Politikken
\end{itemize}\bigskip 

Sentralt ang. \textit{Aristoteles}:
\begin{itemize}\bigskip
    \item Levende vesener er substanser med ein natur
    \item Ville gjenreise fornuften som ein metode for å oppnå kunnskap om denne verden.
    \item Empirist (\ref{empirisme})
    \item Opptatt av naturen og forskning
    \item Tanker og ideer i bevisstheita er kome av sanseinntrykk (syn/hørsel)
    \item Form og stoff 
    \item Tre former for lykke
    \item Meinte at kvinna var ein "uferdig mann"
    
\end{itemize}\bigskip


\bigskip
\section{René Descartes}
\label{descartes}\bigskip
\begin{center}
    \textit{"Cogito ergo sum"}\bigskip    
\end{center}


Ynskte at kunnskapen skulle vera sikkker. Av dette utvikla han teorier som skulle sørga for at ein oppnådde sikker kunnskap. Såleis var det hovudsakleg epistomologiske spørsmål han jobba med.

\bigskip
Litteratur:
\begin{itemize}
    \item Metafysikken
    \item Meditasjoner over filosofiens grunnlag
    \item Om metoden
\end{itemize}\bigskip 

Sentralt ang. \textit{Descartes}:
\begin{itemize}\bigskip
    \item Kartesisk tvil
    \item Religiøs, Gud eksisterer
    \item Menneske har fri vilje
\end{itemize}\bigskip

\section{David Hume}
\label{hume}\bigskip

Ein av dei fremste empiriske filosofane. Han arbeidde hovudsakleg innan epistemologien. Humes erkjennelsesteorie eg også viktig å få med seg. Alle forestillinger me har, må ha blitt bygd på inntrykk me har fått, dvs. ha si opprinning i eitt (eller fleire) inntrykk.
\bigskip

Litteratur:
\begin{itemize}\bigskip
    \item Ein samanfatning av ei nyleg utgitt bok, kalla Ein avhandling om menneskenaturen, osv
    \item Ein avhandling om menneskenaturen
\end{itemize}\bigskip 

Sentralt ang. \textit{Hume}:
\begin{itemize}\bigskip
    \item Empirist (\ref{empirisme})
    \item Fellow feeling
    \item Følelse og tru
\end{itemize}\bigskip


\section{Immanuel Kant}
\label{kant}\bigskip

Tysk filosof, 1724-1804. Hovudinteresser var epistomologi, etikk, metafysikk og logikk. Hans ideer til filosofien var t.d \textbf{Det katagoriske imperativ} og \textbf{transcendental idealisme}.\bigskip

Litteratur:
\begin{itemize}
    \item Kritikk av den rene fornuft
    \item Grunnlegging til moralens metafysikk
    \item Moralens metafysikk
    \item Antropologi i pragmatisk perspektiv
\end{itemize}\bigskip 

Sentralt ang. \textit{Kant}:
\begin{itemize}\bigskip
    \item Pliktetikar
    \item Erkjennelse består av tenking og sansing
\end{itemize}\bigskip

\section{Simone de Beauvoir}
\label{beauvoir}\bigskip

Fransk filosof og forfattar. 1908-1986. Feminist. 1949; Le deuxième Sexe (det annet kjønn). 
\bigskip

Sentralt ang. \textit{Beauvoir}:
\begin{itemize}\bigskip
     \item feminist
\end{itemize}\bigskip

\bigskip
\section{Relativisme}
\label{relativisme}
Er motstand mot tanken om at vitenskapene kan og bør tilstrebe objektivt. Relativistene meinar at det ikkje fins standardar for objektivitet, sannhet eller rasjonalitet som er felles på tvers av forskjellige kulturer og historiske perioder.
\bigskip

Meiningsrelativisme\bigskip
\begin{itemize}
    \item Forenlig med absolutisme
    \item Det fins ulike meiningar om handlingstyper som er enten rett eller galt: Det som er rett for ein person, trenger ikkje å vera rett for ein annan person.
\end{itemize}\bigskip

Konvensjonalisme\bigskip
\begin{itemize}
    \item Forenlig med absolutisme
    \item Ein handling er moralsk rett viss og berre viss den er tillatt ifølge samfunnets konvensjoner
\end{itemize}\bigskip

Begrepsrelativisme\bigskip
\begin{itemize}
    \item Har ikkje nokko klart svar på kva som gjer ein handling rett, me må samanlikna. Kva som gjer ein handling rett er relativt til personens interesser, verdie og det moralske systemet han trur på.
\end{itemize}

\bigskip
\section{Utilitarisme}
\label{utilitarisme}

Utilitarismen er moralfilosofi, ein filosofi for moralske problemstillinger, som prøver å formulera etiske og sosiale problemstillingar på ein slik måte at det tilfredstiller empiriske krav.
\bigskip

Utilitarismen kort oppsumert:\bigskip
\begin{itemize}
    \item Kva som avgjer om ei handling eller ei beslutning frå eit moralsk synspunkt er god eller ikkje, er kva for konsekvenser den har.
    \item Det er viktig når me skal vurdere konsekvensene som er dårlege eller ikkje, at ein tar med i kva for grad av lykke eller ulykke dette medfører.
    \item Når me skal vurdere summen av lykke eller ulykke så tel alle sansende, levende vesener likt. 
\end{itemize}\bigskip

Regelutilitarisme i motsetnad til utilitarisme; Eit visst regelsett du ikkje skal overgå.
\bigskip

\bigskip
\section{Deontolgi}
\label{deontologi}

Deontologi er ein teori som seier at val skal baseres på regler eller moralske prinsipper, og at der 'riktiga valet' veg tyngre enn resultatane.

Deontologi er pliktetikk. I dette ligg det at det må finnes abolutte regler me vil vera pliktige å overhalda. 
\section{Dydsetikk}
\label{dydsetikk}\bigskip

Dydsetikk, ein form for moralteori som baserer seg på begrepet om dyd.\newline

Det fins ulike varianter av dydsetikk, men dei har alle saman til felles at begrepet om dyd speler ei sentral eller uavhengig rolle i å besvare moralske spørsmål. Dydsetikken er ein av dei tre leiande teoriane innan normativ etikk. Store delar av den moderne dydsetikken har røtter i etikken til Aristoteles.\bigskip

Dei to andre leiande teoriane innan normativ etikk, deontologi og utilitarisme, har begge til felles at dei har som hovedformål å formulera allmenne prinsipper som er meint å bestemma kva som er riktige og gale handlinger. Dei er med andre ord ute etter prinsippielle kriterier som me kan anvenda i bestemte situasjoner for å utleie riktig handling.\bigskip

Dydsetikk fokuserer derimot på den handlendes karakter heller enn reglar for riktig atferd. Dydsetikken vil typisk meina at det ikkje lar seg gjera å firmulera nokon allmenngyldig regel som kan ta hensyn til alle forhold som kan vera moralsk viktig i dei enkelte situasjoner. Gjennom erfaring er det mogleg å opparbeida seg ei evne til følsomhet ovanfor korleis ulike forhold kan vera moralsk relevant i ulike situasjoner.\bigskip

Dydsetikara vil også meina at dei skil seg frå deontologer og utilitaristar ved at dydsetikken ikkje berre er ut etter kva som utgjer ei riktig handling. Moralske vurderinger må fokusera vel så mykje på korlei ein person bør vera, og på kva holdninger som ligger tuk grunn for ein persons atferd. \bigskip

\section{Foster og dyrs moralske status, miljøetikk}
\label{fdms}\bigskip

Miljøetikk er den greina av etikken som analyserer og/eller begrunner verdiene og dei normative reglane som underligger menneskers forhold til miljøet, dvs. til den levande og ikkje-levande naturen.\bigskip

Det er mange forskjellige retningar i miljøetikken. Dette kan t.d vera deontologi (pliktetikk) eller konsekvensialistisk (konsekvensetikk). 

\section{Feministisk etikk}
\label{femimisme}\bigskip

Feministisk etikk baserer seg på eit sterkt ynskje om å få slutt på undertrykkelsen, utnyttelsen og mishandling av kvinner, kvar enn dette måtte føregå. Spesielt pornografi og prostitusjon har og er viktige saker for feministane. 

\section{Rasjonalisme}
\label{rasjonalisme}\bigskip

\textit{Rasjonalisme, det syn at fornuften og tenkningen, ikkje sansene, er dei verkelige kunnskapskjeldene}

\section{Empirisme}
\label{empirisme}\bigskip

\textit{Empirisme, erfaringslæra, erfaringsfilosofi; læra om at alle påstandar om verkelegheita skal ha sitt grunnlag i erfaringen og at den er vår eineste kunnskapskjelde. Motsatt: rasjonalisme. Viss erfaringen tenkes utelukkande som sansing, kalles empirismen for sensualisme.}

\end{document}
